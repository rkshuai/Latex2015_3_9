\chapter{绪论}
\label{chap:quickstart}

\section{研究背景与意义}
虚拟现实作为一种新兴技术正快速的改变着人类对物体的认知过程和方式。其以沉浸性、交互性、构想性(3I)而成为目前非常活跃的一个技术领域。它为人机交互的发展开创了一种全新的研究领域,并在航空航天、建筑设计、影视娱乐、医学和教育等方面都起着非常广泛的作用。它通过对用户视觉、听觉、触觉等感官的模拟,使用户仿佛置身于虚拟世界之中,并可以与虚拟环境进行主动的交互。而虚拟人手的构建与实时交互作为虚拟现实技术的一个子课题得到越来越多学者的关注。

科学研究表明,人手是人身体最重要的三大器官之一,我们在生活中进行的大量操作都是直接或间接通过人手完成的。它在我们的日常生活中扮演着非常重要的角色,并且它是人类在视觉方面最重要的组成部分。通过人手我们可以方便的感知物体、抓取物体从而让我们更加准确和快速的完成复杂的任务。模拟和控制虚拟人手是虚拟现实中一个非常重要的课题,在虚拟现实中虚拟人手扮演着与真实人手同样的角色(比如\cite{Wang13})。

通常我们都是使用键盘、鼠标或者手持式装置去与计算机进行交互,这些设备都比较笨重并且不稳定。而虚拟人手为我们提供了一个更加具有前景的替代方案,通过虚拟人手我们能够更加直观和方便的控制计算机。因此目前虚拟人手被广泛应用于手语教学、三维游戏,人机交互以及医学实验等领域。如果能够模拟出一个逼真的虚拟人手将会为我们带来更加强烈的视觉冲击,从而让我们能够在虚拟世界得到更好的体验。

虚拟人手的逼真性和准确性直接影响了仿真分析结果的可靠性以及用户的体验度。但是人手是一个非常复杂的结构,整个人手由27根骨骼组成,约占人身体骨骼的四分之一,并且由大量的血管、神经、肌腱以及韧带等组成,因此想要模拟出一个逼真人手变得非常的困难。随着计算机硬件水平的提高,在虚拟现实应用无处不在的今天,模拟出一个非常逼真的人手无论在理论研究还是产业化应用方面都具有非常重大的意义。

\section{研究发展现状}

尽管人手在我们日常生活中随处可见,但是因为其巨大的复杂性,在计算机图形学应用中人手的模拟并未得到广泛的关注,尤其是在国内并未得到众多学者的研究。但是由于人手的重要性以及科技水平的不断提高,想必在不久的未来这种现状将会得到改善。

想要构建出一个逼近真实的虚拟人手,第一步需要做的就是建模。目前的建模技术主要有3D扫描,三维软件建模以及基于图像等几种方法。3D扫描仪以其高精度的优势而得到大量用户的关注,但是因受成本、空间的限制而只被一些企业所应用。基于图像的三维建模技术是通过拍摄一组采样图像并通过拼接融合等方法将二维图像恢复成景物的三维几何结构的技术,如Apple公司推出的QuickTime VR系统,以及Autodesk公司推出的Autodesk 123D软件,但是因为技术的限制通过这种方法生成的3维模型还不能精确的反映出物体的结构。而目前使用最多的是通过三维软件直接建模的方式,如目前被广泛使用的3ds
max, maya以及Auto CAD等软件,通过这些软件用户可以直接使用点和线来构建出想要的三维模型,这种建模技术因其成本低,简单实用而得到广泛应用。

虚拟人手最重要的目的是为了让用户与计算机进行交互,这方面的研究依然在激烈的进行着。在\cite{Jieun06}中,Jieun
Lee等人提出一种新的方法可以对虚拟人手进行逼真的变形,并通过碰撞检测与自碰撞检测技术实时的与物体进行交互。Paul
G.Kry等人则是采用EigenSkin的方法允许虚拟人手进行细微的非线性准静态变形\cite{Kry02}。
考虑到真实人手的运动学和生物力学特征,Mamy
Pouliquen等人在\cite{Mamy06}提出了一种基于物理的人手模型从而能够方便的通过运动捕捉系统对虚拟人手进行操控。
Matei
Ciocarlie等人则在\cite{Mate07}使用局部几何特性为手指与物体之间的接触建立了一种分析模型。考虑到接触力与物体的形变\cite{Tongcui08,Kchui02},虚拟人手被用来抓取弹性物体。在3D虚拟仿真技术中虚拟手套是最常用的工具,通过虚拟手套我们能够方便的与虚拟环境进行实时的交互\cite{Huagen09,Chin05,Wenzhen11,Hanan12}。

虽然目前已经有一些虚拟人手的研究在进行,但是一个栩栩如生的虚拟人手尚未充分开发并使用。如有些文献只是采用骨骼来去模拟虚拟手,更有一些实验简单的采用小球或者其他物体来代替人手。这很难让用户在虚拟世界中产生沉浸感,同时也降低了许多虚拟动作的准确度,比如抓取等。如果我们能够使用和用户看起来相同的个性化虚拟人手而不是一个通用的人手,这将会为我们在虚拟环境中漫游带来更好的体验。

普遍认为人手具有27个自由度之多,并且人手的构造非常复杂。手指之间的相关性,骨骼控制的复杂性以及皮肤易变形这些因素都严重的制约着一个逼真和精确的虚拟人手的构建。Irene
Albrecht等人\cite{Irene03}与Taehyun
Rhee等人\cite{Rhee06}均采用一张拍摄得到的人手图片去对通用人手进行变形从而得到个性化人手,但是他们并未实现纹理映射的自动化。纹理是物体最重要的特征之一,不同的人手通常具有完全不同的纹理,如果对虚拟人手进行纹理的渲染,我们将会很容易根据纹理去分辨不同的人手,并能够为我们在虚拟世界中带来更加强烈的视觉冲击。但是对不规则物体进行纹理映射同样面临着很多困难。我们通常需要计算三维物体的空间位置并且使用一个或多个映射函数将其映射到纹理空间中,然后使用纹理空间值从纹理中获得相应的值\cite{Tomas02}。另外如果直接采用拍摄得到的不同位置的纹理图对虚拟人手进行渲染时,由于光照以及拍摄角度等因素,将会在两幅图像的连接处产生明显的接缝,Kun
Zhou等人提出了一种纹理融合技术可以解决这种问题\cite{Kun05}。

在虚拟人手与用户的交互方面,虚拟人手的实时交互性能并不高,大部分都仅仅是采用预定义的姿态或者是使用昂贵并且笨重的数据手套来完成。这样显然降低了虚拟手与虚拟环境的交互性。

\section{论文的主要工作和章节安排}
本论文提出了一种通过对一般人手进行自动纹理贴图从而生成个性化人手的方法并可以让虚拟人手实时的与虚拟环境进行交互。其具体步骤为:首先,拍摄真实人手的正面和背面图,并利用拍摄得到的背面图调整一般人手模型使其和背面图可以相重合。因为拍摄时可能由于拍摄角度以及手的晃动等问题造成拍摄得到的正面图并不能与背面图的镜像图完全重合,从而在直接进行纹理贴图时可能会贴到背景,因此我们需要根据背面图的镜像图调整正面图使他们能够完全重合。其次,由于拍摄时的光照问题可能使得拍摄得到的正面图和背面图在颜色上有较大差异,这里我们采用Reinhard等人\cite{Reinhard01}提出的颜色转换方法对正面图进行处理使其可以和背面图颜色相近。再次,分别提取出3D手模型的正面点和背面点,并计算每个点对应到图像上的位置,从而对3D手模型进行贴图。并对两张图的接缝处进行平滑着色处理,使得接缝得到平滑过度。最后,我们为3D手模型添加骨骼,并对骨骼进行蒙皮处理,从而让虚拟人手能够与虚拟环境进行实时交互。

本文的组织结构如下:

第一章介绍了本论文的研究背景、意义以及目前国内外的研究现状。

第二章简要介绍了人手的解剖学结构以及真实人手在运动上的限制,从而可以避免虚拟人手在与虚拟环境的交互过程中产生一些非自然的手势。。

第三章是本文的重中之重,其详细介绍了通过拍摄得到的手心图和手背图对3D手模型进行自动纹理贴图的方法。

第四章介绍了如何通过对3D手模型添加骨骼从而完成虚拟人手与虚拟环境的实时交互,对比了拍摄的真实人手图片以及具有同样姿态的虚拟人手,同时展示了虚拟人手与虚拟环境的交互。

第五章作为本文的最后一个章节,详细的总结了本文的研究工作,指出了本文的优势以及不足,并提出了未来需要做的改进以及研究方向。
