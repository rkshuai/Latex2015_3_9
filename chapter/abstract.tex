\begin{abstract}
虚拟人手在人机交互应用中扮演着非常重要的作用。通过虚拟人手我们能够在虚拟环境中得到更好的体验。但是人手是一个非常复杂的结构,模拟一个逼真的3D虚拟人手并不容易,尤其是为其创建一个灵活的皮肤纹理。现有的虚拟人手的制作都是采用一般的3D模型,并不具有纹理信息。本文提出了一种新的方法可以自动的对3D虚拟人手进行纹理贴图,并能够将其实时的应用到人机交互当中,比如与虚拟环境进行实时交互。

本文的具体做法是首先根据拍摄得到的背面图像调整人手模型,并调整正面图像使其能够能够与背面图像的镜像相重合。之后需要对正面图像进行色彩传递使其可以和背面图像的颜色相似。通过选定裁剪框,我们使用正交投影建立映射函数,从而得到对应纹理坐标处的颜色值。如果直接使用两张图像进行纹理贴图则会在手指的边缘产生明显的接缝,因此我们需要使用平滑着色去除接缝。通过为人手模型添加骨骼结构,可以使其能够像真实人手一样做出各种姿态,从而与虚拟环境进行实时交互。本文提出的方法非常有效,实验结果表明生成的个性化人手比传统的虚拟人手更加逼真,为我们在虚拟环境中带来更强的视觉冲击感。

\keywords{人机交互,虚拟人手,纹理贴图,实时交互}
\end{abstract}

\begin{englishabstract}
Virtual hand plays an important role in many human-computer interaction
applications. We can get a better experience in a virtual environment
through a virtual hand. However, hand is a very complex structure, modeling
a life-like 3D virtual hand is not a trivial task, especially the creation
of flexible skin textures. The existing virtual hand production are based on the
general hand of the 3D models, without the texture information. In this paper, we propose a new method to execute texture mapping for the 3D
virtual hand model automatically and apply it for human-computer
interaction in real-time. 

Specifically, We first adjust the palm image
using the mirror of the dorsal image. Then color transfer is performed for
the palm image to ensure its similar color as the dorsal image. By
selecting the crop box, we can establish the mapping function using orthogonal
projection, the colors can be getted from the texture coordinates. Finally
smooth shading is used to further adjust the direct texture mapping considering the virtual seams in fingers and thumb side. The proposed method is very efficient  and experimental results show that the generated virtual hand model is more realistic than conventional virtual hands and can yield a stronger sense of immersion in the virtual world.

\englishkeywords{human-computer interaction,virtual hand model,texture-mapping,real-time interactivity}
\end{englishabstract}
