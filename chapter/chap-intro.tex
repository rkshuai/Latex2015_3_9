\chapter{手部解剖结构}
\label{chap:intro}
人手作为人身体的三大器官之一,在人的日常生活中起着非常重要的作用。据了解,在哺乳类动物中,人类的手是独一无二的。比如人手的大拇指可以同其他手指对合,但是和人类生理结构最为接近的类人猿却做不到,因为他们的手指还不够柔韧\cite{handBaike}。能够完美自如的运用自己的手指是人类科技和文化进步的关键所在。

为了模拟出逼真的人手模型,我们需要对人手的解剖结构进行深入的研究。同时为了避免在虚拟人手在与虚拟环境进行实时交互的过程中产生一些非自然的手势,我们还需要研究手指与手指之间在运动上的限制关系。

\section{骨骼结构}

人手是一个非常复杂的结构,它是由大量的骨骼、血管、神经、肌腱和韧带组成。因此人的手可以完成很多复杂和灵巧的动作,比如捏、握、抓等。人手是一个多关节系统,共包含27根骨骼\cite{Tubiana90},分别为8根腕骨,5根掌骨和14根指骨组成。两只手加起来的骨骼数目约占人身体骨骼的四分之一。一个骨骼结构如图1所示。


\begin{figure}[htb]
\centering
\includegraphics[width=2.8in, height=2.2in]{../img/035.jpg}
\caption{手部骨骼结构}
\label{fig:graph}
\end{figure}

通常底部的8块腕骨在手的运动过程中只有微小的形状变化\cite{gong2005},因此在对虚拟人手进行操控时,我们并不过度关注虚拟人手腕骨的运动。但是需要注意的是整个骨骼的坐标空间都是以腕骨骼为根节点进行的。掌骨的近侧端称为底,并与腕骨相连。而远侧端称为头,并与指骨相连。其中大拇指的掌骨较短较粗,但是比别的掌骨都更为灵活。指骨作用最大,我们通常进行的抓、捏等动作都是通过指骨来完成的。因此我们重点来研究指骨。

人手的食指、中指、无名指和小拇指都有三根指骨,分别是远指骨(distal
phalanges)、中指骨(middle phalanges)和近指骨(proximal
phalanges)\cite{Kapit93}。每个指骨都是通过关节与别的指骨或掌骨进行相连,这些关节分别为远指骨关节(distal
interphalangeal(DIP))、近指骨关节(proximal
interphalangeal(PIP))和掌指关节(metacarpophalangeal(MCP))。较为特殊的大拇指则只有两根指骨而没有中指骨,但它依然具有三个关节,这三个关节分别是普通关节(IP)、掌指关节(MCP)和腕掌关节(CMC)。

人手之所以那么灵活是因为其自由度很高。其中远指骨关节、近指骨关节以及大拇指的普通关节只在屈伸方向上具有一个自由度,而掌指关节不只在屈伸方向上有一个自由度,他们在内收和外展方向上也具有一个自由度,因此认为这些关节具有两个自由度。另外由于大拇指腕掌关节强大的运动能力,因此可以认为腕掌关节具有两个自由度。最后因为人手在空间中的运动我们认为它还具有另外六个自由度,因此人手总共具有27个自由度。图2为一个简略的人手自由度分布图:

\begin{figure}[htb]
\centering
\includegraphics[width=2.8in, height=2.2in]{../img/DOF.jpg}
\caption{人手自由度分布图}
\label{fig:graph}
\end{figure}


\section{手指间运动的约束性}

手指之间因为一些运动上的限制而不能做出任意的手势。比如当手指的运动超出一定的范围时,其能弯曲的角度必然会受到其他手指的限制。另外,手指并不能向背面过度弯曲。虽然有些非自然手势可以通过重复性的练习而做出来,但这种特殊情况在本章中并不加以考虑。我们主要研究人类手指运动的一般性情况。

另外虽然有一些手指之间的运动关系可以通过大量的研究来进行量化分析与定义,但有一些行为却很难从数学角度上进行分析,这些行为是人们约定俗成的一种观念。比如当我们由伸展开的手掌变成拳头时,通常会同时弯曲所有手指而不是逐个弯曲每根手指。再比如我们想用任意远指骨触碰手掌掌心时,在不弯曲相邻手指的前提下我们是难以达到这一要求的。虽然这些行为可能并不适合于所有人,但是它们却适用于绝大多数的人群。尽管无法量化这些行为,但我们仍然需要遵循这些行为。

首先是人手指关节之间的限制,通常我们认为这存在两种形式的限制。

第一种是单根手指上关节的限制\cite{Beifang05},这种限制通常存在于远指关节与近指关节之间,当我们想活动远指关节时必然要用到与之相邻的近指关节,否则远指关节将很难单独运动。因此我们认为远指关节与近指关节存在一定的相关性。假设远指关节弯曲的角度为$\theta_{DIP}$,而近指关节随之弯曲的角度为$\theta_{PIP}$,这两个角度之间存在如下的线性关系:

\begin{equation}
  \theta_{DIP}=\frac{2}{3}\theta_{PIP}
\end{equation}

另一种是相邻手指关节的限制。这种限制通常存在于掌指关节之间。比如当我们弯曲中指的掌指关节时,超过一定范围后,若想继续弯曲,则必然会带动其相邻食指和无名指的掌指关节才能得到更大程度的弯曲。这些限制我们暂时还不能从公式上进行明确定义。

其次是人手指在未借助外力的情况下所能弯曲的角度范围,我们能够使用下面的不等式来量化他们\cite{John00}:

\begin{equation}
 %\label{eq:barSE}
 \begin{split}
  \theta^{f}_{MCP\_FE} & \in(-10^\circ, 90^\circ) \\
  \theta^{f}_{PIP\_FE} & \in(~~~0^\circ, 110^\circ) \\
  \theta^{f}_{DIP\_FE} & \in(~~~0^\circ, 60^\circ) \\
  \theta^{f}_{MCP\_AA} & \in(-15^\circ, 15^\circ) \\
  \theta^{t}_{MCP\_FE} & \in(~~~0^\circ, 90^\circ) \\
  \theta^{t}_{IP\_FE}  & \in(-10^\circ, 80^\circ) \\
  \theta^{t}_{MCP\_AA} & \in(-30^\circ, 30^\circ) 
 \end{split}
\end{equation}
其中$FE$代表屈和伸(flexion/extension),而$AA$代表内收和外展(abduction/adduction),下标$t$代表大拇指,而$f$代表除大拇指外的其他手指。

另外对于中指来说通常具有很小的外展动作,因此我们普遍认为:
\begin{equation}
  \theta_{MCP\_AA}=0^{0}
\end{equation}

手指之间的运动还具有很多约束性,这里我们并不一一讨论,遵循这些约束将使我们生成的虚拟人手更加逼真。

\section{本章小结}
本章主要介绍了人手的骨骼结构以及手指与手指之间的约束性,通过对这些知识的介绍可以避免虚拟人手在与虚拟环境进行实时交互时产生非自然手势,从而为我们将来模拟出逼真的虚拟人手打下坚实的基础。
