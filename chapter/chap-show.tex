\chapter{虚拟手的实时交互}
通过纹理映射后的虚拟人手并不能直接与虚拟环境进行交互,为了使其能够像真是人手一样做出各种姿态,我们可以采用骨骼动画的方式为其添加骨骼从而实时的控制虚拟人手。
\section{骨骼实现人机交互}
当我们为3D人手模型贴上纹理之后,如果只是对整个人手进行旋转平移而不是像真实人手一样能够实现各种手势的话,对于我们的模型本身来说并没有太大的意义。真实人手之所以能够实现抓捏等动作主要是通过骨骼来控制的,而在计算机图形学中,为了让虚拟人手模型能够想真实人手一样,我们依然可以采取对虚拟人手赋予骨骼的方式来实现,不过这里的骨骼不再是真实的骨骼,而是虚拟骨骼。

目前在计算机图形学中为了实现复杂物体的运动主要的方式有两种,一种是帧动画,另一种是骨骼动画。帧动画是通过分解一个动作将其存放在很多帧中,然后对这些帧连续播放就可以形成一个完整的动作。对于每帧动画都需要存取所有的顶点,动作越流畅需要的帧数就越多,因此采用这种方式需要非常大的存储空间。骨骼动画则主要是通过存取骨骼的旋转和平移矩阵,并利用骨骼与骨骼之间的相对关系,从而使得绑定在骨骼上的顶点发生改变。这种方式相对帧动画方式来说虽然增加了一些计算量,但是存储空间得到极大的减少。因此常被用于游戏、动画与影视当中。图4.1描述了一个简易的帧动画和骨骼动画\cite{Pipho02}:

\begin{figure}[htb]
\centering
\includegraphics[width=2.9in, height=1.5in]{../img/guge.jpg}
\caption{帧动画和骨骼动画}
\label{fig:graph}
\end{figure} 

通过上图可以看出帧动画需要调整每一个顶点的位置,而骨骼动画则只需要调整骨骼即可,骨骼可以带动绑定其上的顶点运动到指定位置。

帧动画通常只能通过预先的定义来展示我们所需要的动作,但是我们在运行人手模型时则要使它能够与虚拟环境进行实时的交互,而不能采用预定的行为,骨骼模型可以满足我们实时的要求。综上所述,我们采用骨骼动画的方式来处理我们的人手模型。

3DS
MAX软件具有为3D模型添加骨骼的功能,通过为人手模型赋上骨骼,并采用蒙皮来将顶点绑定在骨骼上。绑定骨骼后的人手模型如图3.33所示:

\begin{figure}[htb]
\centering
\includegraphics[width=2.0in, height=2.35in]{../img/handGuge.jpg}
\caption{骨骼绑定}
\label{fig:graph}
\end{figure} 

从图中可以看出整个骨骼模型只有一个根骨骼,位于腕骨处。人手的运动都是以根骨骼的关节点为原点运行的,当移动根骨骼时整个人手模型都会运动。两根骨骼的连接处叫做节点,一根骨骼由两个节点控制,分别被称为父节点和子节点,其中父节点的运动控制着其下所有节点的运动。比如当食指的掌指关节运动时必然带动其下的近指关节和远指关节同时运动。对于人手模型来说一个节点只有一个父节点与其相连,因此能够简化我们的编程模块。

上面谈论了很多关于骨骼的概念,但是对于编程模型来说骨骼并不是真实存在的,可以将骨骼看成是坐标空间,而节点用来描述骨骼所在的位置,也就是坐标原点。因此使骨骼绕关节旋转就可以看成是坐标空间自身的旋转。每根骨骼都有一个相对于其父骨骼的相对矩阵,这个相对矩阵由旋转和平移操作组成,根骨骼因为没有父骨骼因此其相对矩阵就等于绝对矩阵,也就是在世界坐标系中的位置。通过将骨骼的相对矩阵与其父骨骼的绝对矩阵相乘就可得到每根骨骼的绝对矩阵。而正是因为只需要存储骨骼的相对矩阵,因此骨骼动画非常节省内存。

另外因为3ds模型并不适合进行骨骼处理,因此我们选用ms3d模型,这种模型最初是为经典的$Half-Life$游戏而设计的,因其小巧灵活的特性而成为目前使用率很高的一种模型,使用这种模型能够让我们更方便的处理骨骼动画。

下面我们就介绍顶点跟随骨骼一起运动的方法。

首先将顶点绑定到骨骼之上,并定义当前骨骼的父骨骼所在空间的绝对矩阵为$matrixGlobal\_PAR$,当前骨骼的绝对矩阵为$matrixGlobal\_SON$,当前骨骼的相对矩阵为$matrixLocal$,骨骼变换矩阵为$matrixMotion$。$matrixGlobal\_SON$的计算方式为:
\begin{eqnarray}
  matrixGlobal\_SON=matrixGlobal\_PAR\times matrixLocal
\end{eqnarray}

当我们控制骨骼发生运动时,要想使得顶点跟随骨骼一起运动,则要先将顶点由世界空间变换到骨骼空间中,变换的方式是使顶点坐标右乘当前骨骼绝对矩阵的逆矩阵$matrixGlobal\_SON^{-1}$。而此时当前骨骼的相对矩阵为:
\begin{eqnarray}
  matrixLocal\_NEW=matrixLocal\times matrixMotion
\end{eqnarray}
从而当前骨骼所在空间的绝对矩阵变为:
\begin{eqnarray}
  matrixGlobal\_NEW=matrixGlobal\_PAR\times matrixLocal\_NEW
\end{eqnarray}

将顶点再右乘以$matrixGlobal\_NEW$即可将顶点由骨骼空间变换到世界空间中去,从而实现顶点跟随骨骼的移动。

这里需要注意的是对于手指这种特殊模型的运动,通常只会发生旋转而不会有平移,因此$matrixMotion$只包含旋转矩阵即可。

如果我们将每个顶点只绑定到一个骨骼上面,对于非关节处并不会有问题。但是对于处于关节处的顶点在弯曲度较大的情况下会发生断裂,这会严重影响模拟的逼真度。为解决此问题,我们可以在对骨骼绑定顶点时给骨骼赋予不同的权重,使得处于关节处顶点受到多个骨骼的拉扯作用\cite{ZhangXL03},从而解决骨骼的断裂问题。图4.2(a)和4.2(b)分别显示了未对骨骼赋予权重与对骨骼赋予权重后的效果图。
\begin{figure}[htb]
\begin{tabular}{cc}
~~~~~~~~~~~\includegraphics[width=1.7in, height=1.6in]{../img/duanlie2.jpg}&
~~~~~~~~\includegraphics[width=1.7in,height=1.6in]{../img/weiduanlie3.jpg}\\
~~~~~~~~~~(a)& ~~~~~~~(b)
\end{tabular}
\caption{为骨骼添加权重: (a)添加前; (b)添加后.}
\end{figure}
从图中我们可以清晰的看出处理前在掌指关节处有明显的断裂,但是通过对骨骼添加权重信息,使得不同骨骼同时作用于顶点,非常好的解决了断裂问题。

使用骨骼带动顶点运动虽然改变了顶点位置,但是三角形面片索引和纹理索引并未发生改变,因此纹理能够跟随骨骼一起运动。但是对于影响光照信息的法线则需要重新计算,对顶点法线来说其变化的矩阵即为骨骼变化的矩阵$matrixMotion$,只不过我们只需要考虑旋转变化而不用考虑平移变化。

通过以上这些步骤即可让我们生成一个逼真的虚拟人手,并可与虚拟环境进行实时交互。

\section{实验结果与分析}
虚拟人手的交互主要包括虚拟手在虚拟空间中的运动以及对真实人手动作的模拟\cite{Duhong11}。考虑到数据手套的贵重以及不方穿戴性,我们使用键盘来控制我们的虚拟人手。考虑到linux平台的开源性,我们在linux操作系统下通过C++语言和OpenGL图形库来实现。另外因为我们需要对拍摄得到的图像做一定的处理,因此还用了OpenCV图像库。用于实验的计算机配置了AMD六核CPU,主频3.5GHZ以及4G内存,显卡型号为ATI
Radeon HD 7750。

我们首先拍摄了一组具有各种姿态的真实人手图像,然后控制我们的虚拟人手使其能够做出同样的姿态,其对比图如图4.3所示。另外,为了让我们的手能够与虚拟环境进行交互,我们模拟了虚拟人手与布料以及虚拟人手与球的交互,其效果图如图4.4所示。
\begin{figure}[htb]
\begin{tabular}{cccccc}
\subfigure[]{
\begin{minipage}[c]{0.125\textwidth}
\includegraphics[width=2.3cm]{../img/image0.jpg}
\end{minipage}} &
\subfigure[]{
\begin{minipage}[c]{0.125\textwidth}
\includegraphics[width=2.3cm]{../img/image1t.jpg}
\end{minipage}} &
\subfigure[]{
\begin{minipage}[c]{0.125\textwidth}
\includegraphics[width=2.3cm]{../img/image2t.jpg}
\end{minipage}}& 
\subfigure[]{
\begin{minipage}[c]{0.125\textwidth}
\includegraphics[width=2.3cm]{../img/image3t.jpg}%
\end{minipage}} &
\subfigure[]{
\begin{minipage}[c]{0.125\textwidth}
\includegraphics[width=2.3cm]{../img/image4t.jpg}
\end{minipage}} &
\subfigure[]{
\begin{minipage}[c]{0.125\textwidth}
\includegraphics[width=2.3cm]{../img/image5t.jpg}
\end{minipage}}\\ 
\subfigure[]{
\begin{minipage}[c]{0.125\textwidth}
\includegraphics[width=2.8cm]{../img/modelOver0.jpg}
\end{minipage}} &
\subfigure[]{
\begin{minipage}[c]{0.125\textwidth}
\includegraphics[width=2.8cm]{../img/modelOver1.jpg}
\end{minipage}} &
\subfigure[]{
\begin{minipage}[c]{0.125\textwidth}
\includegraphics[width=2.8cm]{../img/modelOver2.jpg}
\end{minipage}}& 
\subfigure[]{
\begin{minipage}[c]{0.125\textwidth}
\includegraphics[width=2.8cm]{../img/modelOver3.jpg}
\end{minipage}} &
\subfigure[]{
\begin{minipage}[c]{0.125\textwidth}
\includegraphics[width=2.8cm]{../img/modelOver4t.jpg}
\end{minipage}} &
\subfigure[]{
\begin{minipage}[c]{0.125\textwidth}
\includegraphics[width=2.8cm]{../img/modelOver5.jpg}
\end{minipage}}\\ 
\end{tabular}
\caption{真实人手与虚拟人手的对比: (a)-(f)是真实人手的姿态;
(g)-(l)是具有同样姿态的虚拟人手.}
\end{figure}

\begin{figure}[htb]
\begin{tabular}{ccc}
\includegraphics[width=4.5cm, height=2.5cm]{img/model_cloth.jpg}&
\includegraphics[width=4.5cm, height=2.5cm]{img/model_ball.jpg}&
\includegraphics[width=4.5cm, height=2.5cm]{img/model_ball1.jpg}\\
(a)& (b) & (c)
\end{tabular}
\caption{虚拟人手与虚拟环境的实时交互: (a) 抓取布料; (b) 握住圆球; (c)
捏住圆球.}
\end{figure}


从图中可以看出经过纹理贴图后的虚拟人手极大的增强了模拟的逼真性,使我们能够在虚拟环境中获得更好的体验。因为虚拟人手的数据通常是轻量级的,因此能够让我们快速的与虚拟环境进行实时交互。

\section{本章小结}
为了使得经过纹理贴图后的虚拟人手能够与虚拟环境进行交互,本章介绍了一种骨骼动画的方法,通过为虚拟人手添加骨骼,并采用蒙皮等技术,可以让虚拟人手像真实人手一样做出各种姿态,能够与虚拟环境实时交互,从而为我们带来更好的体验。
