\chapter{工作总结与展望}

\section{论文工作总结}
本文主要研究了虚拟人手的纹理贴图以及与虚拟环境的实时交互。虚拟人手在游戏、3D动画、手语学习中都扮演着非常重要的作用。但是如果我们只是采用一般的人手模型并不能为我们带来很好的视觉体验,通过对人手进行纹理贴图则可以为我们生成逼真的个性化人手,从而使我们能够虚拟环境获得更好的体验。

我们首先描述了人手的骨骼结构以及手指与手指之间的约束性,通过对这些知识的介绍可以避免我们所生成的虚拟人手在与虚拟环境进行实时交互时产生非自然手势。纹理是物体重要的表面特征之一,为了生成个性化的虚拟人手,我们需要对人手进行纹理贴图,因此拍摄两张真实人手的纹理图像。通常我们需要先根据背面人手图像进行模型的调整,因为拍摄的原因正面人手图像并不能直接贴在模型的正面,因此需要对正面图像进行调整使其能够与背面图像的镜像重合。因为两张图像的色彩会有差异,因此需要对正面图像进行色彩传递使其可以和背面图像的色彩相似。为了使得两张图像贴到模型对应的位置,需要提取模型的正反面顶点。之后设定裁剪框,并采用正交投影从而为我们建立映射函数,根据映射函数即可找到每个顶点所对应的纹理坐标。为人手贴上纹理后会在两张纹理的连接出产生明显的虚拟接缝,利用平滑着色中的Gouraud着色法可以去除接缝。为了使得经过纹理贴图后的个性化人手能够进行抓捏等操作,我们可以采用骨骼动画的方法为3D人手模型赋上骨骼,利用蒙皮等技术使得虚拟人手能够像真实人手一样做出各种姿态,并能够与虚拟环境进行实时交互。

\section{研究展望}
由于时间和个人知识水平的限制,本文的研究工作仍有地方需要进一步的改进和提高:

(1)在通过背面人手图像对模型进行调整时,我们是用手工的方式进行调节的。这部分工作其实可以采用基于径向基函数的方式对模型进行自动调节,目前这块工作已经开展了一部分。

(2)真实人手在进行抓捏等操作时肌肉会有明显的变形,比如折叠等,而这些变化并不能反应在我们的虚拟人手上,下一步的工作需要将人手的生物特性考虑进去。
